\documentclass[onecolumn]{ctexart}
\usepackage[utf8]{inputenc}
\usepackage{amsmath}
\usepackage{amssymb}
\usepackage{amsthm}
\usepackage{mathtools}
\usepackage{geometry}
\usepackage{graphicx}
\usepackage{float}
\usepackage{xcolor}
\usepackage{listings}
\usepackage{indentfirst}
\usepackage{bm}
\usepackage{tikz}
\usetikzlibrary{shapes,arrows}
\geometry{a4paper,scale=0.8}

\newtheorem{definition}{Definition}
\newtheorem{theorem}{Theorem}
\newtheorem{proposition}{Proposition}
\newtheorem{lemma}{Lemma}
\newtheorem{corollary}{Corollary}
\newtheorem{remark}{Remark}
\newtheorem{example}{Example}

\title{Notes of "Elementary Facts about Series"}
\author{Jinxin Wang}
\date{}

\begin{document}

\maketitle

\section{Overview}
\begin{itemize}
  \item The sum of a series and the Cauchy criterion for convergence of a series
  \begin{itemize}
    \item Def: A (infinite) series
    \item Def: The terms of a series and the n-th term
    \item Def: The (n-th) partial sum of a series
    \item Def: A convergent or divergent series
    \item Def: The sum of a series
    \item Thm: The Cauchy convergence criterion for a series
    \item Cor: The equivalence of the convergence between two series with only finite terms different
    \item Cor: A necessary condition for a series to be convergent in terms of the limit of its terms
  \end{itemize}
  \item Absolute convergence, the comparison theorem and its consequences
  \begin{itemize}
    \item Def: A series is absolutely convergent
    \item Rmk: Absolute convergence implies convergence, but the opposite is not true
    \item Thm: Criterion for convergence of series of nonnegative terms in terms of bounds
    \item Thm: (Comparison theorem) Test for convergence of series of nonnegative terms in terms of comparison with another nonnegative series
    \item Cor: (The Weierstrass M-test for absolute convergence)
    \item Cor: (Cauchy's test)
    \item Cor: (d'Alembert's test)
    \item Prop: (Cauchy) A necessary and sufficient condition for a monotonic nonnegative series to be absolutely convergent in terms of the generated series $\Sigma_{k=0}^{\infty}2^ka_{2^k}$
    \item Cor: The convergence of the series $\zeta(p) = \Sigma_{n=1}^{\infty} \frac{1}{n^p}$
  \end{itemize}
  \item The number $e$ as the sum of a series
\end{itemize}

\section{The sum of a series and the Cauchy criterion for convergence of a series}

\section{Absolute convergence, the comparison theorem and its consequences}

\begin{definition}[A series is absolutely convergent]
  The series $\Sigma_{n=1}^{\infty} a_n$ is absolutely convergent if the series $\Sigma_{n=1}^{\infty} \vert a_n \vert$ is convergent.
\end{definition}

\begin{remark}[Absolute convergence implies convergence, but the opposite is not true]
  
\end{remark}

\begin{theorem}[Criterion for convergence of series of nonnegative terms in terms of bounds]
  
\end{theorem}

\begin{theorem}[(Comparison theorem) Test for convergence of series of nonnegative terms in terms of comparison with another nonnegative series]
  
\end{theorem}

\begin{corollary}[(The Weierstrass M-test for absolute convergence)]
  Let $\Sigma_{n=1}^{\infty} a_n$ and $\Sigma_{n=1}^{\infty} b_n$ be series. Suppose there exists an index $N \in \mathbb{N}$ such that $\vert a_n \vert \leq b_n$ for all $n > N$. Then a sufficient condition for absolute condition of the series $\Sigma_{n=1}^{\infty} a_n$ is that the series $\Sigma_{n=1}^{\infty} b_n$ converge.
\end{corollary}

\section{The number $e$ as the sum of a series}

\end{document}