\documentclass[onecolumn]{ctexart}
\usepackage[utf8]{inputenc}
\usepackage{amsmath}
\usepackage{amssymb}
\usepackage{amsthm}
\usepackage{mathtools}
\usepackage{geometry}
\usepackage{graphicx}
\usepackage{float}
\usepackage{xcolor}
\usepackage{listings}
\usepackage{indentfirst}
\usepackage{bm}
\usepackage{tikz}
\usetikzlibrary{shapes,arrows}
\geometry{a4paper,scale=0.8}

\newtheorem{definition}{Definition}
\newtheorem{theorem}{Theorem}
\newtheorem{proposition}{Proposition}
\newtheorem{lemma}{Lemma}
\newtheorem{corollary}{Corollary}
\newtheorem{remark}{Remark}
\newtheorem{example}{Example}

\title{Notes of "Definition and Properties of the Limit of a Sequence"}
\author{Jinxin Wang}
\date{}

\begin{document}

\maketitle

\section{Overview}

\section{Definition of the Limit of a Sequence}

\begin{definition}[A Sequence]
  A function $f: \mathbb{N} \to X$ is called a sequence.
\end{definition}

\begin{definition}[The Limit of a Numerical Sequence]
  A number $A \in \mathbb{R}$ is called the limit of the numerical sequence 
  $\lbrace x_n \rbrace$ if for every neighborhood $V(A)$ of $A$ there exists an 
  index $N$ (depending on $V(A)$) such that all terms of the sequence having 
  index larger than N belong to the neighborhood $V(A)$.
\end{definition}
\begin{remark}[An Equivalent Definition of the Limit of a Numerical Sequence with $\epsilon-N$]
  Another equivalent definition of the limit of a numerical sequence is:

  A number $A \in \mathbb{R}$ is called the limit of the numerical sequence 
  $\lbrace x_n \rbrace$ if for every $\epsilon > 0$ there exists an index $N$ 
  (depending on $\epsilon$) such that $|x_n - A| < \epsilon$ for all $n > N$.

  Proof of equivalence: (TODO)
\end{remark}
\begin{remark}[Formulation of definitions of the limit of a numerical sequence in symbolic logic]
  \[
    (\lim_{n \to \infty} x_n = A) \coloneqq (\forall V(A) \exists N \in \mathbb{N} \forall n > N (x_n \in V(A)))
  \]
  \[
    (\lim_{n \to \infty} x_n = A) \coloneqq (\forall \epsilon > 0 \exists N \in \mathbb{N} \forall n > N (|x_n - A| < \epsilon))
  \]
\end{remark}

\begin{definition}[A convergent/divergent sequence]
  If $\lim_{n \to \infty} x_n = A$, we say that the sequence $\lbrace x_n 
  \rbrace$ converges to $A$ or tends to $A$ and write $x_n \to A$ as $n \to 
  \infty$.

  A sequence having a limit is said to be convergent. A sequence that does not 
  have a limit is said to be divergent.
\end{definition}

\section{Properties of the Limit of a Sequence}

\subsection{General Properties}

The word "general" means that the properties in this section are possessed not 
only by numerical sequences, but by other kinds of sequences as well.

\begin{definition}[An ultimately constant sequence]
  If there exists a number $A$ and an index $N$ such that $x_n = A$ for all $n > 
  N$, the sequence $\lbrace x_n \rbrace$ will be called ultimately constant.
\end{definition}

\begin{definition}[A bounded sequence]
  A sequence $\lbrace x_n \rbrace$ is bounded if there exists $M$ such that 
  $|x_n| < M$ for all $n \in \mathbb{N}$.
\end{definition}

\begin{theorem}
  \begin{description}
    \item[T1] An ultimately constant sequence converges.
    \item[T2] Any neighborhood of the limit of a sequence contains all but a 
    finite number of terms of the sequence.
    \item[T3] A convergent sequence cannot have two different limits.
    \item[T4] A convergent sequence is bounded.
  \end{description}
\end{theorem}
\begin{proof}
  \begin{description}
    \item[Pf of T1] 
  \end{description}
\end{proof}

\subsection{Properties Involving Arithmetic Operations}

\begin{definition}[The sum, product and quotient of two numerical sequences]
  If $\lbrace x_n \rbrace$ and $\lbrace y_n \rbrace$ are two numerical sequences, 
  their sum, product, and quotient are the sequences
  \[
    \lbrace (x_n + y_n) \rbrace, \lbrace (x_n \cdot y_n) \rbrace, \lbrace (\frac{x_n}{y_n}) \rbrace
  \]
  The quotient is defined only when $y_n \neq 0$ for all $n \in \mathbb{N}$.
\end{definition}

\begin{theorem}[]
  Let $\lbrace x_n \rbrace$ and $\lbrace y_n \rbrace$ be numerical sequences. If 
  $\lim_{n \to \infty} x_n = A$ and $\lim_{n \to \infty} y_n = B$, then
  \begin{description}
    \item[T1] $\lim_{n \to \infty} (x_n + y_n) = A + B$
    \item[T2] $\lim_{n \to \infty} (x_n \cdot y_n) = AB$
    \item[T3] $\lim_{n \to \infty} (\frac{x_n}{y_n}) = \frac{A}{B}$, provided 
    $y_n \neq 0 (n=1,2,\ldots)$ and $B \neq 0$
  \end{description}
\end{theorem}
\begin{proof}
  \begin{description}
    \item[Pf of T1] 
    \item[Pf of T2] 
    \item[Pf of T3] 
  \end{description}
\end{proof}

\subsection{Properties Involving Inequalities}

\begin{theorem}
  \begin{description}
    \item[T1] (保序性) Let $\lbrace x_n \rbrace$ and $\lbrace y_n \rbrace$ be two 
    convergent sequences with $\lim_{n \to \infty} x_n = A$ and $\lim_{n \to 
    \infty} y_n = B$. If $A < B$, then there exists an index $N \in \mathbb{N}$ 
    such that $x_n < y_n$ for all $n > N$.
    \item[T2] (夹逼性) Suppose the sequences $\lbrace x_n \rbrace$, $\lbrace y_n \rbrace$, 
    and $\lbrace z_n \rbrace$ are such that $x_n \leq y_n \leq z_n$ for all $n > 
    N \in \mathbb{N}$. If the sequences $\lbrace x_n \rbrace$ and $\lbrace z_n 
    \rbrace$ both converge to the same limit, then the sequence $\lbrace y_n 
    \rbrace$ also converges to that limit.
  \end{description}
\end{theorem}
\begin{proof}
  \begin{description}
    \item[Pf of T1] 
    \item[Pf of T2] 
  \end{description}
\end{proof}

\begin{corollary}
  Suppose $\lim_{n \to \infty} x_n = A$ and $\lim_{n \to \infty} y_n = B$. If 
  there exists $N$ such that for all $n > N$ we have
  \begin{description}
    \item[C1] $x_n > y_n$, then $A \geq B$.
    \item[C2] $x_n \geq y_n$, then $A \geq B$.
    \item[C3] $x_n > B$, then $A \geq B$.
    \item[C4] $x_n \geq B$, then $A \geq B$.
  \end{description}
\end{corollary}
\begin{remark}
  Notice that strict inequality in terms may become equality in the limit. 
  Example: $\frac{1}{n} > 0$ for all $n \in \mathbb{N}$ but $\lim_{n \to \infty} 
  \frac{1}{n} = 0$.
\end{remark}

\section{Infinity}

\subsection{Definition of Infinity}

\begin{definition}[Infinity, Positive Infinity, Negative Infinity]
  
\end{definition}

\begin{corollary}[Relation between Infinity and Infinitesimal]
  
\end{corollary}

\begin{definition}[Not an Infinity]
  
\end{definition}

\subsection{Operations Involving Infinity}



\end{document}