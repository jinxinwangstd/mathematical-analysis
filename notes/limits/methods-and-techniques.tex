\documentclass[onecolumn]{ctexart}
\usepackage[utf8]{inputenc}
\usepackage{amsmath}
\usepackage{amssymb}
\usepackage{amsthm}
\usepackage{mathtools}
\usepackage{geometry}
\usepackage{graphicx}
\usepackage{float}
\usepackage{xcolor}
\usepackage{listings}
\usepackage{indentfirst}
\usepackage{bm}
\usepackage{tikz}
\usetikzlibrary{shapes,arrows}
\geometry{a4paper,scale=0.8}

\newtheorem{definition}{Definition}
\newtheorem{theorem}{Theorem}
\newtheorem{proposition}{Proposition}
\newtheorem{lemma}{Lemma}
\newtheorem{corollary}{Corollary}
\newtheorem{remark}{Remark}
\newtheorem{example}{Example}

\title{Notes of "Methods and Techniques in Limits"}
\author{Jinxin Wang}
\date{}

\begin{document}

\maketitle

\section{Overview}

\section{Definition and Properties of the Limit of a Sequence}

\subsection{数列放缩技巧}
\begin{itemize}
  \item 加一项减一项结合三角不等式
\end{itemize}

\section{Existence of the Limit of a Sequence}

\subsection{求数列极限}
\begin{itemize}
  \item 定义法:使用的前提是已知极限值。
  \item 夹逼法
  \item Cauchy Proposition, Stolz Theorem, Topelitz Theorem
  \item 单调有界数列收敛原理
  \item Cauchy's Convergence Criterion
\end{itemize}

\subsection{判定数列发散}
\begin{itemize}
  \item 利用数列发散的定义(数列收敛定义的否命题)
  \item 利用收敛数列性质的逆否命题
  \begin{itemize}
    \item 无界数列一定发散
    \item 有两个收敛到不同极限值的子列的数列一定发散
  \end{itemize}
  \item 考察子列特性: 有一个发散子列的数列一定发散
  \item Cauchy's Convergence Criterion
\end{itemize}

\subsection{考察数列单调性}

\end{document}