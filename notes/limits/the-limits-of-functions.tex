\documentclass[onecolumn]{ctexart}
\usepackage[utf8]{inputenc}
\usepackage{amsmath}
\usepackage{amssymb}
\usepackage{amsthm}
\usepackage{mathtools}
\usepackage{geometry}
\usepackage{graphicx}
\usepackage{float}
\usepackage{xcolor}
\usepackage{listings}
\usepackage{indentfirst}
\usepackage{bm}
\usepackage{tikz}
\usetikzlibrary{shapes,arrows}
\geometry{a4paper,scale=0.8}

\newtheorem{definition}{Definition}
\newtheorem{theorem}{Theorem}
\newtheorem{proposition}{Proposition}
\newtheorem{lemma}{Lemma}
\newtheorem{corollary}{Corollary}
\newtheorem{remark}{Remark}
\newtheorem{example}{Example}

\DeclareMathOperator{\sgn}{sgn}

\title{Notes of "The Limit of A Function"}
\author{Jinxin Wang}
\date{}

\begin{document}

\maketitle

\section{Definitions and Examples}

\begin{definition}[The Limit of a Function (Basic Type)]
  ($\epsilon$-$\delta$)

  \begin{equation}
    \forall \epsilon > 0, \exists \delta > 0, \forall x \in E (0 < |x - a| < \delta \Rightarrow |f(x) - A| < \epsilon)
  \end{equation}
\end{definition}

\begin{remark}
  Based on the definition, we can see that limits are a kind of local 
  characteristic of a function. With that, when using the definition to prove 
  the limit of a function at $x = x_0$, we can discuss it within a certain 
  neighborhood $O(x_0, \delta_0)$.
\end{remark}

\begin{definition}[A deleted neighborhood of a point]
  A deleted neighborhood of a point is a neighborhood of the point from which 
  the point itself has been removed.
\end{definition}

\begin{definition}[函数极限的邻域定义]
  \begin{equation}
    (\lim_{E \owns x \to a} f(x) = A) := \forall U_{\mathbb{R}}(A) \exists \dot{U}_E(a) (f(\dot{U}_E(a)) \subset U_{\mathbb{R}}(A))
  \end{equation}
\end{definition}
\begin{remark}
  If $f$ is convergent at $a$, $a$ must be a limit point of its domain $E$.
\end{remark}

\begin{example}[符号函数$\sgn x$]
  
\end{example}

\begin{proposition}[Heine's Proposition]
  
\end{proposition}

\begin{corollary}[Existence of the Limit of a Function by Limits of Sequences]
  
\end{corollary}

\section{Properties of the Limit of a Function}

\begin{remark}
  In order to establish the properties of the limit of a function, we need only 
  two properties of deleted neighborhoods of a limit point of a set:
  \begin{itemize}
    \item $\dot{U}_E(a) \neq \emptyset$
    \item $\forall \dot{U}_E^1(a) \dot{U}_E^2(a) \exists \dot{U}_E(a) 
    (\dot{U}_E(a) \subset \dot{U}_E^1(a) \cap \dot{U}_E^2(a))$
  \end{itemize}
  This observation leads us to a general concept of a limit of a function and 
  the possibility of using the theory of limits in the future not only for 
  functions defined on sets of numbers.
\end{remark}

\subsection{General Properties}

\begin{definition}[最终常数函数]
  
\end{definition}

\begin{definition}[有界函数/最终有界函数,上有界函数/最终上有界函数,下有界函数/最终下有界函数]
  A function $f: E \to \mathbb{R}$ is bounded, bounded above, or bounded below 
  respectively if there is a number $C \in \mathbb{R}$ such that $|f(x)| < C$, 
  $f(x) < C$, or $C < f(x)$ for all $x \in E$.

  If one of these three relations holds only in some deleted neighborhood 
  $\dot{U}_E(a)$, the function is said to be ultimately bounded, ultimately 
  bounded above, or ultimatedly bounded below as $E \owns x \to a$ respectively.
\end{definition}

\begin{theorem}
  \begin{description}
    \item[$1^o$] (Ultimate Constant has the Limit) $(f:E \to \mathbb{R}$ is ultimately the constant $A$ as $E 
    \owns x \to a)$ $\Rightarrow$ $\lim_{E \owns x \to a} f(x) = A$
    \item[$2^o$] (Ultimately Boundness of the Limit) $(\exists \lim_{E \owns x \to a} f(x))$ $\Rightarrow$ $(f: E \to \mathbb{R}$ is ultimately bounded as $E \owns x \to a)$
    \item[$3^o$] (Uniqueness of the Limit) $(\lim_{E \owns x \to a} f(x) = A_1) \wedge (\lim_{E \owns x \to a} f(x) = A_2)$ $\Rightarrow$ $A_1 = A_2$
  \end{description}
\end{theorem}
\begin{proof}
  
\end{proof}

\subsection{Properties Involving Arithmetic Operations}

\begin{definition}[两个函数的和、积与商]
  
\end{definition}

\begin{theorem}[四则运算中的函数极限]
  Let $f:E \to \mathbb{R}$ and $g:E \to \mathbb{R}$ be two functions with a 
  common domain of definition. If $\lim_{E \owns x \to a} f(x) = A$ and 
  $\lim_{E \owns x \to a} g(x) = B$, then
  \begin{itemize}
    \item $\lim_{E \owns x \to a} f(x) + g(x) = A + B$
    \item $\lim_{E \owns x \to a} f(x) g(x) = AB$
    \item $\lim_{E \owns x \to a} \frac{f(x)}{g(x)} = \frac{A}{B}$, if 
    $B \neq 0$ and $g(x) \neq 0$ for $x \in E$.
  \end{itemize}
\end{theorem}

\begin{definition}[Infinitesimal]
  A function $f: E \to \mathbb{R}$ is said to be infinitesimal as $E \owns x \to 
  a$ if $\lim_{E \owns x \to a} f(x) = 0$
\end{definition}

\begin{proposition}[无穷小函数的四则运算性质]
  Let $\alpha: E \to \mathbb{R}$ is an infinitesimal as $E \owns x \to a$, then
  \begin{itemize}
    \item If $\beta: E \to \mathbb{R}$ is also an infinitesimal as $E \owns x 
    \to a$, then their sum $\alpha + \beta$ is also an infinitesimal as $E \owns 
    x \to a$.
    \item If $\beta: E \to \mathbb{R}$ is also an infinitesimal as $E \owns x 
    \to a$, then their product $\alpha \beta$ is also an infinitesimal as $E 
    \owns x \to a$.
    \item If $\beta: E \to \mathbb{R}$ is ultimately bounded as $E \owns x \to a$,
    then their product $\alpha \beta$ is also an infinitesimal as $E \owns x \to 
    a$. 
  \end{itemize}
\end{proposition}
\begin{proof}
  
\end{proof}
\begin{remark}
  \begin{equation}
    (\lim_{E \owns x \to a} f(x) = A) \Leftrightarrow (\lim_{E \owns x \to a} (f(x) - A) = 0)
  \end{equation}
\end{remark}

\subsection{Properties Involving Inequalities}
\begin{theorem}
  \begin{description}
    \item[$1^o$] (局部保序性) If the functions $f: E \to \mathbb{R}$ and $g: E \to \mathbb{R}$ are such that 
    \item[$2^o$] (夹逼性)
  \end{description}
\end{theorem}

\begin{corollary}[局部保号性]
  
\end{corollary}

\begin{corollary}[极限值的不等关系]
  Suppose $\lim_{E \owns x \to a} f(x) = A$ and $\lim_{E \owns x \to a} g(x) = 
  B$. Let $\dot{U}_E(a)$ be a deleted neighborhood of $a \in E$.
  \begin{description}
    \item[(a)] If $f(x) > g(x)$ for all $x \in \dot{U}_E(a)$, then $A \geq B$.
    \item[(b)] If $f(x) \geq g(x)$ for all $x \in \dot{U}_E(a)$, then $A \geq B$.
    \item[(c)] If $f(x) > B$ for all $x \in \dot{U}_E(a)$, then $A \geq B$.
    \item[(d)] If $f(x) \geq B$ for all $x \in \dot{U}_E(a)$, then $A \geq B$.
  \end{description}
\end{corollary}

\subsection{Two Important Examples}

\begin{example}[研究$\frac{\sin x}{x}$在$x=0$处的极限]
  \begin{equation}
    \lim_{x \to 0} \frac{\sin x}{x} = 1
  \end{equation}
\end{example}

\begin{example}[定义指数函数、对数函数和幂函数]
  
\end{example}

\section{The General Definition of the Limit of a Function}

\subsection{Definition and Examples of a Base}

\begin{definition}
  A set $\mathcal{B}$ of subsets $B \subset X$ of a set $X$ is called a base in 
  $X$ if the following conditions hold:
  \begin{itemize}
    \item $\forall B \in \mathcal{B}$ $(B \neq \emptyset)$
    \item $\forall B_1, B_2 \in \mathcal{B} \exists B \in \mathcal{B}$ $(B 
    \subset B_1 \cap B_2)$
  \end{itemize}
\end{definition}

Some useful bases in analysis
\[
  \begin{matrix}
    x \to a & \dot{U}(a):= \lbrace x \in \mathbb{R} \mid a - \delta_1 < x < a + \delta_2 \wedge x \neq a, \delta_1 > 0, \delta_2 > 0\rbrace \\
    x \to a+0 & E_a^+ := \lbrace x \in \mathbb{R} \mid a < x \rbrace, \dot{U}_{E_a^+}(a) := E_a^+ \cap \dot{U}(a) = \lbrace x \in \mathbb{R} \mid a < x < a + \delta, \delta > 0 \rbrace \\
    x \to a-0 & E_a^+ := \lbrace x \in \mathbb{R} \mid x < a \rbrace, \dot{U}_{E_a^-}(a) := E_a^- \cap \dot{U}(a) = \lbrace x \in \mathbb{R} \mid a - \delta < x < a, \delta > 0 \rbrace \\
    x \to \infty & U(\infty) := \lbrace x \in \mathbb{R} \mid |x| > \delta, \delta \in \mathbb{R} \rbrace \\
    x \to +\infty & E_{\infty}^+ := \lbrace x \in \mathbb{R} \mid c < x, c \in \mathbb{R} \rbrace, U_{E_{\infty}^+}(\infty) := E_{\infty}^+ \cap U(\infty) = \lbrace x \in \mathbb{R} \mid c < x, c \in \mathbb{R} \rbrace \\
    x \to -\infty & E_{\infty}^- := \lbrace x \in \mathbb{R} \mid x < c, c \in \mathbb{R} \rbrace, U_{E_{\infty}^-}(\infty) := E_{\infty}^- \cap U(\infty) = \lbrace x \in \mathbb{R} \mid x < c, c \in \mathbb{R} \rbrace \\
    E \owns x \to a & \dot{U}_E(a) := E \cap \dot{U}(a) \\
    E \owns x \to a+0 & \dot{U}_E(a+0) := E \cap E_a^+ \cap \dot{U}(a) \\
    E \owns x \to a-0 & \dot{U}_E(a-0) := E \cap E_a^- \cap \dot{U}(a) \\
    E \owns x \to \infty & U_E(\infty) := E \cap U(\infty) \\
    E \owns x \to +\infty & U_E(+\infty) := E \cap E_{\infty}^+ \cap U(\infty) \\
    E \owns x \to -\infty & U_E(-\infty) := E \cap E_{\infty}^- \cap U(\infty)\\
  \end{matrix}
\]

\subsection{Limit over a Base}

\begin{definition}[The Limit of a Function over a Base]
  \begin{equation}
    (\lim_{\mathcal{B}} f(x) = A) := (\forall U(A) \exists B \in \mathcal{B} (f(B) \subset U(A)))
  \end{equation}
\end{definition}

\[
  \lim_{x \to a} f(x) = A := (\forall \epsilon > 0, \exists \delta > 0, \forall x \in (a - \delta, a + \delta): (|f(x) - A| < \epsilon))
\]
\[
  \lim_{x \to a+0} f(x) = A 
\]
\[
  \lim_{x \to a-0} f(x) = A  
\]
\[
  \lim_{x \to \infty} f(x) = A  
\]
\[
  \lim_{x \to +\infty} f(x) = A  
\]
\[
  \lim_{x \to -\infty} f(x) = A  
\]
\[
  \lim_{x \to a} f(x) = \infty
\]
\[
  \lim_{x \to a+0} f(x) = \infty
\]
\[
  \lim_{x \to a-0} f(x) = \infty
\]
\[
  \lim_{x \to \infty} f(x) = \infty
\]
\[
  \lim_{x \to +\infty} f(x) = \infty
\]
\[
  \lim_{x \to -\infty} f(x) = \infty
\]
\[
  \lim_{x \to a} f(x) = +\infty
\]
\[
  \lim_{x \to a+0} f(x) = +\infty
\]
\[
  \lim_{x \to a-0} f(x) = +\infty
\]
\[
  \lim_{x \to \infty} f(x) = +\infty
\]
\[
  \lim_{x \to +\infty} f(x) = +\infty
\]
\[
  \lim_{x \to -\infty} f(x) = +\infty
\]
\[
  \lim_{x \to a} f(x) = -\infty
\]
\[
  \lim_{x \to a+0} f(x) = -\infty
\]
\[
  \lim_{x \to a-0} f(x) = -\infty
\]
\[
  \lim_{x \to \infty} f(x) = -\infty
\]
\[
  \lim_{x \to +\infty} f(x) = -\infty
\]
\[
  \lim_{x \to -\infty} f(x) = -\infty
\]

\section{The Existence of the Limit of a Function}

\subsection{Cauchy's Criterion}

\begin{definition}[Oscillation]
  
\end{definition}

\begin{theorem}[Cauchy's Criterion on the Limit of a Function]
  
\end{theorem}

\subsection{The Limit of a Composite Function}

\begin{theorem}[Theorem of the Limit of a Composite Function]
  
\end{theorem}

\begin{example}
  \begin{equation}
    \lim_{x \to \infty} (1 + \frac{1}{x})^x = e
  \end{equation}
\end{example}

\subsection{The Limit of a Monotonic Function}

\begin{definition}[Monotonic Functions]
  A function $f:E \to \mathbb{R}$ defined on a set $E \subset \mathbb{R}$ is 
  said to be
  \begin{itemize}
    \item increasing if $\forall x_1, x_2 \in E (x_1 < x_2 \Rightarrow f(x_1) < f(x_2))$
    \item nondecreasing if $\forall x_1, x_2 \in E (x_1 < x_2 \Rightarrow f(x_1) \leq f(x_2))$
    \item decreasing if $\forall x_1, x_2 \in E (x_1 < x_2 \Rightarrow f(x_1) > f(x_2))$
    \item nonincreasing if $\forall x_1, x_2 \in E (x_1 < x_2 \Rightarrow f(x_1) \geq f(x_2))$
  \end{itemize}
\end{definition}
\begin{remark}
  Be mindful of that the monotonicity is defined on the complete domain of the 
  definition, and thus the condition is $\forall x_1, x_2 \in E$ no matter what 
  a set $E$ is.
\end{remark}

\begin{theorem}[The Existence Criterion of the Limit of a Monotonic Function]
  
\end{theorem}

\subsection{Comparison of the Limiting Behaviors of Functions}

\begin{proposition}
  
\end{proposition}

\begin{remark}
  The above proposition can not be generalized to the limit of a sum of 
  functions.
\end{remark}

\section{方法与技巧}

\subsection{证明与研究函数极限}

\begin{itemize}
  \item 定义法
  \item 变量代换
\end{itemize}

\begin{remark}
  在研究函数极限时使用变量代换是否总是成立?这个问题使用数学语言来描述如下:\\
  Suppose that $\lim_{x \to a} g(x) = A$, $\lim_{x \to A} f(x) = B$, is it true that 
  \[
    \lim_{x \to a} f(g(x)) = \lim_{y \to A} f(y)
  \]
  In fact it is not always true. Here is a counterexample: Let $g(x) \equiv 0$, 
  and thus $\lim_{x \to 0} g(x) = 0$. Let $f(x) = 
  \begin{cases}
    1, x = 0 \\
    0, x \neq 0
  \end{cases}$. Then we have $\lim_{x \to 0} f(x) = 0$, $\lim_{x \to 0} f(g(x)) 
  = 1$. 这里的数学直观是$\lim_{x \to a} g(x) = A$决定了$y = g(x) \to A$的方式。它与
  $\lim_{x \to A} f(x) = B$中$x \to A$的不同可能导致结果的不同。

  Here are two propositions related to this problem:
  \begin{proposition}
    Suppose that $\lim_{x \to a} g(x) = A$, $\lim_{x \to A} f(x) = B$. If any of 
    the following conditions is true:
    \begin{itemize}
      \item $\exists \delta_0 > 0$ such that $\forall x \in O(a, \delta_0) 
      \setminus \{a\}$: $g(x) \neq A$.
      \item $\lim_{x \to A} f(x) = f(A)$.
      \item $A = \infty$, and $\lim_{x \to \infty} f(x)$ is defined.
    \end{itemize}
    then the following is true:
    \[
      \lim_{x \to a} f(g(x)) = \lim_{y \to A} f(y)
    \]
  \end{proposition}
  \begin{proof}
    Hint:
    \begin{itemize}
      \item Notice the difference between the conclusion of the definition of 
      $\lim_{x \to a} g(x) = A$, and the condition of the definition of 
      $\lim_{x \to A} f(x) = B$.
      \item Notice what change the fact of continuity brings to the definition 
      of $\lim_{x \to A} f(x) = B$.
    \end{itemize}
  \end{proof}

  \begin{proposition}
    If $\lim_{x \to a} g(x) = A$, $\lim_{x \to A} f(x) = B$, then exact one of 
    the following situation is true:
    \begin{itemize}
      \item $\lim_{x \to a} f(g(x)) = B$
      \item $\lim_{x \to a} f(g(x)) = g(A)$
      \item $\lim_{x \to a} f(g(x))$ is not defined
    \end{itemize}
  \end{proposition}
  \begin{proof}
    Hint: Using the first condition of the previous proposition to discuss 
    different kinds of $g(x)$.
  \end{proof}
\end{remark}

\end{document}