\documentclass[onecolumn]{ctexart}
\usepackage[utf8]{inputenc}
\usepackage{amsmath}
\usepackage{amssymb}
\usepackage{amsthm}
\usepackage{mathtools}
\usepackage{geometry}
\usepackage{graphicx}
\usepackage{float}
\usepackage{xcolor}
\usepackage{listings}
\usepackage{indentfirst}
\usepackage{bm}
\usepackage{tikz}
\usetikzlibrary{shapes,arrows}
\geometry{a4paper,scale=0.8}

\newtheorem{definition}{Definition}
\newtheorem{theorem}{Theorem}
\newtheorem{proposition}{Proposition}
\newtheorem{lemma}{Lemma}
\newtheorem{corollary}{Corollary}
\newtheorem{remark}{Remark}
\newtheorem{example}{Example}

\title{Notes of "Existence of the Limit of a Sequence"}
\author{Jinxin Wang}
\date{}

\begin{document}

\maketitle

\section{Overview}

\begin{itemize}
  \item The Cauchy Criterion
  \begin{itemize}
    \item Definition: A fundamental or Cauchy sequence
    \item Theorem: Cauchy's convergence criterion
  \end{itemize}
  \item A Criterion for the Existence of the Limit of a Monotonic Sequence
  \begin{itemize}
    \item Definition: An increasing/nondecreasing/nonincreasing/decreasing sequence and monotonic sequences
    \item Definition: A bounded-above sequence
    \item Theorem: Weierstrass's theorem
  \end{itemize}
\end{itemize}

\section{The Cauchy Criterion}

\begin{definition}[A fundamental or Cauchy sequence]
  A sequence $\lbrace x_n \rbrace$ is called a fundamental or Cauchy sequence if 
  for any $\epsilon > 0$ there exists an index $N \in \mathbb{N}$ such that 
  $|x_m - x_n| < \epsilon$ whenever $n > N$ and $m > N$.
\end{definition}

\begin{theorem}[Cauchy's convergence criterion]
  A numerical sequence converges if and only if it is a Cauchy sequence.
\end{theorem}

\begin{example}
  \[
    x_n = 1 + \frac{1}{2} + \cdots + \frac{1}{n}
  \]

  \[
    \begin{split}
      x_{2n} - x_{n} &= \frac{1}{n+1} + \frac{1}{n+2} + \cdots + \frac{1}{2n} \\
                     &> n \cdot \frac{1}{2n} \\
                     &= \frac{1}{2}
    \end{split}
  \]
\end{example}

\section{A Criterion for the Existence of the Limit of a Monotonic Sequence}

\begin{definition}[An increasing/nondecreasing/nonincreasing/decreasing sequence and monotonic \\ sequences]
  A sequence $\lbrace x_n \rbrace$ is
  \begin{itemize}
    \item increasing if $x_n < x_{n+1}$ for all $n \in \mathbb{N}$
    \item nondecreasing if $x_n \leq x_{n+1}$ for all $n \in \mathbb{N}$
    \item nonincreasing if $x_n \geq x_{n+1}$ for all $n \in \mathbb{N}$
    \item decreasing if $x_n > x_{n+1}$ for all $n \in \mathbb{N}$
  \end{itemize}

  Sequences of these four types are called monotonic sequences.
\end{definition}

\begin{definition}[A bounded-above sequence]
  
\end{definition}

\begin{theorem}
  In order for a nondecreasing sequence to have a limit it is necessary and 
  sufficient that it be bounded above.
\end{theorem}
\begin{proof}
  
\end{proof}
\begin{remark}
  An analogous theorem exists that it is sufficient and necessary for a 
  nonincreasing sequence to have a limit that it be bounded below.
\end{remark}

\begin{example}
  $\lim_{n \to \infty} \frac{n}{q^n} = 0$ if $q > 1$.
\end{example}

\begin{corollary}
  \[
    \lim_{n \to \infty} \sqrt[n]{n} = 1
  \]
\end{corollary}

\begin{corollary}
  \[
    \lim_{n \to \infty} \sqrt[n]{a} = 1 \thickspace \textnormal{for any} \thickspace a > 0
  \]
\end{corollary}

\begin{example}
  $\lim_{n \to \infty} \frac{q^n}{n!} = 0$ where $q \in \mathbb{R}$ and $n \in \mathbb{N}$.
\end{example}

\section{The Number e}

\begin{proposition}
  The sequences $a_n = (1 + \frac{1}{n})^n$ and $b_n = (1 + \frac{1}{n})^{n+1}$ 
  are convergent, and they have the same limit values.
\end{proposition}
\begin{proof}
  
\end{proof}

\begin{definition}
  \[
    e \coloneqq \lim_{n \to \infty} (1 + \frac{1}{n})^n
  \]
\end{definition}

\begin{proposition}
  $e = \Sigma_{n=0}^{\infty} \frac{1}{n!} = 1 + 1 + \frac{1}{2!} + \frac{1}{3!} + \cdots + \frac{1}{n!} + \cdots$
\end{proposition}
\begin{proof}
  
\end{proof}

\begin{proposition}
  $e = \lim_{n \to \infty} \frac{n}{\sqrt[n]{n!}}$
\end{proposition}
\begin{proof}
  
\end{proof}

\section{Subsequences and the Partial Limits}

\begin{definition}[A subsequence of a sequence]
  If $x_1, x_2, \ldots, x_n, \ldots$ is a sequence and $n_1 < n_2 < \cdots < n_k 
  < \cdots$ is an increasing sequence of natural numbers, then the sequence 
  $x_{n_1}, x_{n_2}, \ldots, x_{n_k}, \ldots$ is called a subsequence of the 
  sequence $\lbrace x_n \rbrace$.
\end{definition}

\begin{lemma}[Bolzano-Weierstrass theorem]
  Every bounded sequence of real numbers contains a convergent subsequence.
\end{lemma}
\begin{proof}
  Hint:
  \begin{itemize}
    \item Let $E$ be the set of values of $x_n$. Hence $E$ is bounded.
    \item If the number of elements of $E$ is finite, then there exists $c \in E$ 
    such that $x_{n_1} = x_{n_2} = \cdots = x_{n_k} = \cdots = c$. Hence the 
    subsequence $\lbrace x_{n_1}, x_{n_2}, \ldots, x_{n_k}, \ldots \rbrace$ 
    converges to $c$.
    \item If the number of elements of $E$ is infinite, then by 
    Bolzano-Weierstrass principle it has a limit point $c$.
  \end{itemize}
\end{proof}

\begin{definition}[A sequence tends to positive infinity]
  We shall write $x_n \to +\infty$ and say that the sequence $\lbrace x_n 
  \rbrace$ tends to positive infinity if for each number $c$ there exists $N \in 
  \mathbb{N}$ such that $x_n > c$ for all $n > N$.
\end{definition}
\begin{remark}[Definitions of a sequence tends to negative infinity and tends to infinity]
  There are two analogous definitions of a sequence tends to negative infinity 
  and tends to infinity:
  \begin{itemize}
    \item The sequence $\lbrace x_n \rbrace$ tends to negative infinity if for 
    each number $c$ there exists $N \in \mathbb{N}$ such that $x_n < c$ for all 
    $n > N$.
    \item The sequence $\lbrace x_n \rbrace$ tends to infinity if for each number 
    $c$ there exists $N \in \mathbb{N}$ such that $|x_n| > c$ for all $n > N$.
  \end{itemize}
\end{remark}
\begin{remark}[Definitions of a sequence tends to positive infinity, negative infinity, and infinity in symbolic logic]
  \[
    (x_n \to +\infty) \coloneqq (\forall c \in \mathbb{R} \exists N \in \mathbb{N} \forall n > N (c < x_n))
  \]
  \[
    (x_n \to -\infty) \coloneqq (\forall c \in \mathbb{R} \exists N \in \mathbb{N} \forall n > N (x_n < c))
  \]
  \[
    (x_n \to \infty) \coloneqq (\forall c \in \mathbb{R} \exists N \in \mathbb{N} \forall n > N (c < |x_n|))
  \]
\end{remark}

\begin{lemma}
  From each sequence of real numbers one can extract either a convergent 
  subsequence or a subsequence that tends to infinity.
\end{lemma}
\begin{proof}
  Hint:
  \begin{itemize}
    \item If the sequence is bounded, then by Bolzano-Weierstrass theorem we can 
    extract a convergent subsequence.
    \item If the sequence is unbounded, then for any $c \in \mathbb{R}$, there 
    exists at least one term $|x_k| > c$. So we can construct such a subsequence 
    where the $k$-th term $x_{n_k}$ holds that $|x_{n_k}| > k$ for all $k \in 
    \mathbb{N}$. It is clear that such a subsequence exists and it tends to 
    infinity.
  \end{itemize}
\end{proof}

\begin{definition}[The inferior limit of a sequence]
  The number $l = \lim_{n \to \infty} \inf_{k \geq n} x_k$ is called the inferior 
  limit of the sequence $\lbrace x_k \rbrace$ and denoted 
  $\varliminf_{k \to \infty} x_k$ or $\lim \inf_{k \to \infty} x_k$. If $i_n \to 
  +\infty$, it is said that the inferior limit of the sequence equals positive 
  infinity, and we write $\varliminf_{k \to \infty} x_k = +\infty$ or $\lim 
  \inf_{k \to \infty} x_k = +\infty$. If the original sequence $\lbrace x_k 
  \rbrace$ is not bounded below, then we shall have $i_n = \inf_{k \geq n} x_k = 
  -\infty$ for all $n$. In that case we say that the inferior limit of the 
  sequence equals negative infinity and write $\varliminf_{k \to \infty} x_k = 
  -\infty$ or $\lim \inf_{k \to \infty} x_k = -\infty$.
  \[
    \varliminf_{k \to \infty} x_k \coloneqq \lim_{n \to \infty} \inf_{k \geq n} x_k
  \]
\end{definition}

\begin{definition}[The superior limit of a sequence]
  \[
    \varlimsup_{k \to \infty} x_k \coloneqq \lim_{n \to \infty} \sup_{k \geq n} x_k
  \]
\end{definition}

\begin{definition}[A partial limit of a sequence]
  A number (or the symbol $+\infty$ or $-\infty$) is called a partial limit of a 
  sequence, if the sequence contains a subsequence converging to that number.
\end{definition}

\begin{proposition}
  The inferior and superior limit of a bounded sequence are respectively the 
  smallest and largest partial limits of the sequence.
\end{proposition}
\begin{proof}
  
\end{proof}

\begin{remark}
  The Bolzano-Weierstrass Lemma in its restricted formulation follows from the 
  above proposition.
\end{remark}

\begin{proposition}
  For any sequence, the inferior limit is the smallest of its partial limits and 
  the superior limit is the largest of its partial limits.
\end{proposition}

\begin{remark}
  The Bolzano-Weierstrass Lemma in its wider formulation follows from the above 
  proposition.
\end{remark}

\begin{corollary}
  A sequence has a limit or tends to negative or positive infinity if and only if 
  its inferior and superior limits are the same.
\end{corollary}

\begin{corollary}
  A sequence converges if and only if every subsequence of it converges.
\end{corollary}

\section{The Limit of a Transformed Sequence}

\subsection{Toeplitz's Theorem}

\begin{theorem}
  Suppose there exists a sequence $\{t_{nk}\}$ such that 
  $\forall n, k \in \mathbb{N^+}$, $t_{nk} \geq 0$, $\Sigma_{k=1}^n t_{nk} = 1$, 
  $\lim_{n \to \infty} t_{nk} = 0$. If $\lim_{n \to \infty} a_n = a$, then 
  \[
    \lim_{n \to \infty} \Sigma_{k=1}^n t_{nk} a_k = a
  \]
\end{theorem}

\begin{remark}
  The condition in the Toeplitz' Theorem $\lim_{n \to \infty} t_{nk} = 0$ means 
  that for any given $k$, in other words $k$ is finite, $t_{nk}$ tends to $0$ 
  when $n$ tends to $\infty$. This is supported by the proof, since in the proof 
  we only need the first finite number of terms in the sequence $\{t_{nk}\}$ to 
  converge to $0$.
\end{remark}

\subsection{Stolz's Theorem}

\begin{theorem}[$\frac{0}{0}$ type]
  Suppose $\lim_{n \to \infty} a_n = 0$, $\lim_{n \to \infty} b_n = 0$, and 
  $\{a_n\}$ is decreasing. If 
  \[
    \lim_{n \to \infty} \frac{b_{n+1} - b_n}{a_{n+1} - a_n} = l
  \]
  then 
  \[
    \lim_{n \to \infty} \frac{b_n}{a_n} = l
  \]
\end{theorem}

\begin{theorem}[$\frac{*}{\infty}$ type]
  Suppose $\{a_n\}$ is increasing and $\lim_{n \to \infty} a_n = \infty$. If
  \[
    \lim_{n \to \infty} \frac{b_{n+1} - b_n}{a_{n+1} - a_n} = l
  \]
  then
  \[
    \lim_{n \to \infty} \frac{b_n}{a_n} = l   
  \]
\end{theorem}

\begin{proof}
  Method: By Toeplitz's Theorem
  \[
    t_nk = \{\frac{a_1}{a_n}, \frac{a_2 - a_1}{a_n}, \frac{a_3 - a_2}{a_n}, \cdots, \frac{a_n - a_{n-1}}{a_n}\}
  \]
  \[
    c_n = \{\frac{b_1}{a_1}, \frac{b_2 - b_1}{a_2 - a_1}, \frac{b_3 - b_2}{a_3 - a_2}, \cdots, \frac{b_n - b_{n-1}}{a_n - a_{n-1}}\}
  \]
\end{proof}

\subsection{Cauchy's Proposition}

\begin{proposition}[算术平均值形式]
  If $\lim_{n \to \infty} a_n = a$, then
  \[
    \lim_{n \to \infty} \frac{a_1 + a_2 + \cdots + a_n}{n} = a
  \]
\end{proposition}

\begin{proposition}[算术平均值等价形式]
  If $\lim_{n \to \infty} (a_n - a_{n-1}) = a$, then
  \[
    \lim_{n \to \infty} \frac{a_n}{n} = a
  \]
\end{proposition}

\begin{proposition}[几何平均值形式]
  If $\lim_{n \to \infty} a_n = a > 0$, then
  \[
    \lim_{n \to \infty} \sqrt[n]{a_1 a_2 \cdots a_n} = a
  \]
\end{proposition}

\end{document}