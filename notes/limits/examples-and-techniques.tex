\documentclass[onecolumn]{ctexart}
\usepackage[utf8]{inputenc}
\usepackage{amsmath}
\usepackage{amssymb}
\usepackage{amsthm}
\usepackage{mathtools}
\usepackage{geometry}
\usepackage{graphicx}
\usepackage{float}
\usepackage{xcolor}
\usepackage{listings}
\usepackage{indentfirst}
\usepackage{bm}
\usepackage{tikz}
\usetikzlibrary{shapes,arrows}
\geometry{a4paper,scale=0.8}

\newtheorem{definition}{Definition}
\newtheorem{theorem}{Theorem}
\newtheorem{proposition}{Proposition}
\newtheorem{lemma}{Lemma}
\newtheorem{corollary}{Corollary}
\newtheorem{remark}{Remark}
\newtheorem{example}{Example}

\title{Notes of "Elementary Facts about Series"}
\author{Jinxin Wang}
\date{}

\begin{document}

\maketitle

\section{Overview}

\section{Examples}

\subsection{Series}

\begin{example}[$x^p$-series]
  
\end{example}

\begin{example}[$\ln^p$-series]
  
\end{example}

\begin{example}[Leibniz series]
  
\end{example}

\begin{example}[组级数]
  这类级数的特点是它的项以$k$个为一组循环出现。

  和Leibniz series一样,这类级数可以具有比正项级数更慢的收敛于0的速度,因此常用来构造反例。

  证明其收敛性的思路一般为证明$S_{kn}$(按组求和)收敛,再证明$S_{kn+1}$, $S_{kn+2}$, $\ldots$, $S_{kn+n-1}$收敛到同一个值。

  Eg:
  \[
    \Sigma_{n=1}^{\infty} x_n = \frac{1}{2} + \frac{1}{2} - 1 + \frac{1}{2\sqrt[3]{2}} + \frac{1}{2\sqrt[3]{2}} - \frac{1}{\sqrt[3]{2}} + \cdots + \frac{1}{2\sqrt[3]{k}} + \frac{1}{2\sqrt[3]{k}} - \frac{1}{\sqrt[3]{k}} + \cdots
  \]
  Proof of the convergence:
  \[
    S_{3n} = 0, S_{3n+1} = \frac{1}{2\sqrt[3]{k}}, S_{3n+2} = \frac{1}{\sqrt[3]{k}}
  \]
  \[
    \lim_{n \to \infty} S_{3n} = \lim_{n \to \infty} S_{3n+1} = \lim_{n \to \infty} S_{3n+2} = 0
  \]
  Hence,
  \[
    \Sigma_{n=1}^{\infty} x_n = 0
  \]

  The interesting part of this example is that $\Sigma_{n=1}^{\infty} x_n^3$ is divergent:
  \[
    \begin{split}
      \Sigma_{n=1}^{\infty} x_n^3 &= \frac{1}{8} + \frac{1}{8} - 1 + \frac{1}{16} + \frac{1}{16} - \frac{1}{2} + \cdots + \frac{1}{8k} + \frac{1}{8k} - \frac{1}{k} + \cdots \\
                                  &= \Sigma_{k=1}^{\infty} (-\frac{3}{4k})
    \end{split}
  \]


  Eg:
  \[
    \Sigma_{n=1}^{\infty} x_n = 1 - \frac{1}{2} - \frac{1}{4} + \frac{1}{3} - \frac{1}{6} - \frac{1}{8} + \cdots + \frac{1}{2k - 1} - \frac{1}{4k - 2} - \frac{1}{4k} + \cdots
  \]
  Proof of the convergence:
  \[
    S_{3n} = \Sigma_{k=1}^n \frac{1}{2}(\frac{1}{2k-1} - \frac{1}{2k}), \lim_{n \to \infty} S_{3n} = \frac{1}{2} \Sigma_{n=1}^{\infty} \frac{(-1)^{n+1}}{n} = \frac{\ln2}{2}
  \]
  \[
    S_{3n+1} = S_{3n} + \frac{1}{2k+1}, S_{3n+2} = S_{3n} + \frac{1}{4k+2}
  \]
  \[
    \lim_{n \to \infty} S_{3n+1} = \lim_{n \to \infty} S_{3n+2} = \lim_{n \to \infty} S_{3n} 
  \]

  这个级数是一个$\Sigma_{n=1}^{\infty} \frac{(-1)^{n+1}}{n}$的更序级数,可以看到对于条件收敛的级数,更序级数与它的和可能不相同。因此它可以作为Reimann Theorem的一个例子。
\end{example}

\end{document}