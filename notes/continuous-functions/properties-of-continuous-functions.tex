\documentclass[onecolumn]{ctexart}
\usepackage[utf8]{inputenc}
\usepackage{amsmath}
\usepackage{amssymb}
\usepackage{amsthm}
\usepackage{mathtools}
\usepackage{geometry}
\usepackage{graphicx}
\usepackage{float}
\usepackage{xcolor}
\usepackage{listings}
\usepackage{indentfirst}
\usepackage{bm}
\usepackage{tikz}
\usetikzlibrary{shapes,arrows}
\geometry{a4paper,scale=0.8}

\newtheorem{definition}{Definition}
\newtheorem{theorem}{Theorem}
\newtheorem{proposition}{Proposition}
\newtheorem{lemma}{Lemma}
\newtheorem{corollary}{Corollary}
\newtheorem{remark}{Remark}
\newtheorem{example}{Example}

\title{Notes of "Properties of Continuous Functions"}
\author{Jinxin Wang}
\date{}

\begin{document}

\maketitle

\section{Local Properties}

\begin{theorem}
  \begin{description}
    \item[T1] (局部有界性)
    \item[T2] (局部保号性)
    \item[T3] (四则运算保持连续性)
    \item[T4] (复合函数保持连续性)
  \end{description}
\end{theorem}

\section{Global Properties}

\begin{theorem}[The Bolzano-Cauchy Intermediate-Value Theorem]
  If a function is continuous on a closed interval, and assumes values with 
  opposite signs on the endpoints of the interval, then the function has at 
  least one zero point in the interval.

  In logic expression:
  \[
    (f \in C[a, b]) \wedge (f(a) \cdot f(b) < 0) \Rightarrow \exists c \in (a, b) (f(c) = 0)
  \]
\end{theorem}
\begin{proof}
  Hint: Bisection method.
\end{proof}
\begin{proof}
  Hint: The Lebesgue method.
  \begin{itemize}
    \item Let $A = \lbrace t \mid (a \leq t) \wedge (\forall x \in [a, t] 
    (f(x) < 0)) \rbrace$
    \item The set $A$ is not empty.
    \item $b$ is an upper bound of $A$.
    \item According to the least upper bound principle, there exists $s = \sup A \in [a, b]$.
    \item $f(s) < 0$ is impossible othewise $s$ is not an upper bound of $A$. 
    $f(s) > 0$ is impossible otherwise $s$ is not the least upper bound of $A$.
    Therefore $f(s) = 0$.
  \end{itemize}
\end{proof}
\begin{remark}[Connectivity of the domain]
  
\end{remark}

\begin{corollary}
  If the function $\phi$ is continuous on an open interval and assumes values 
  $\phi(a) = A$ and $\phi(b) = B$ at points $a$ and $b$, then for any number $C$ 
  between $A$ and $B$, there is a point $c$ between $a$ and $b$ at which 
  $\phi(c) = C$.
\end{corollary}
\begin{proof}
  
\end{proof}

\begin{theorem}[The Weierstrass Maximum-Value Theorem]
  If a function is continuous on a closed interval, then it is bounded on the 
  interval. The function assumes the maximum value and minimum value on the 
  interval.
\end{theorem}
\begin{proof}
  Hint: The finite covering lemma
\end{proof}
\begin{remark}[Compactness of the domain]
  
\end{remark}

\subsection{Uniform continuity}

\begin{definition}[Uniform continuity]
  A function $f: E \to \mathbb{R}$ is uniformly continuous on a set $E \subset 
  \mathbb{R}$ if for every $\epsilon > 0$ there exists $\delta > 0$ such that 
  $|f(x_1) - f(x_2)| < \epsilon$ for all points $x_1, x_2 \in E$ such that $|x_1 
  - x_2| < \delta$.

  In logic expression:
  \[
    \begin{split}
      &(f: E \to \mathbb{R} \thickspace \textnormal{is uniformly continuous} \thickspace) \coloneqq \\
      &(\forall \epsilon > 0 \exists \delta > 0 \forall x_1 \in E \forall x_2 \in E (|x_1 - x_2| < \delta \Rightarrow |f(x_1) - f(x_2)| < \epsilon))
    \end{split}
  \]
\end{definition}
\begin{remark}
  If a function is uniformly continuous on a set, then it is continuous on any 
  points in the set.
\end{remark}
\begin{remark}
  In general, a function is continuous on a set cannot derive that the function 
  is uniformly continuous on the set.
\end{remark}
\begin{remark}[The definition of the negation of uniform continuity for a function in logic expression]
  \[
    \begin{split}
      &(f: E \to \mathbb{R} \thickspace \textnormal{is not uniformly continuous} \thickspace) \coloneqq \\
      &(\exists \epsilon > 0 \forall \delta > 0 \exists x_1 \in E \exists x_2 \in E (|x_1 - x_2| < \delta \wedge |f(x_1) - f(x_2)| \geq \epsilon))
    \end{split}
  \] 
\end{remark}
\begin{example}
  $f(x) = \sin\frac{1}{x}$
\end{example}
\begin{remark}[The difference between continuity on a set and uniform continuity on a set]
  In the definition of a function $f: E \to \mathbb{R}$ being continuous on $E$, 
  the number $\delta$ depends on the number $\epsilon$ and the point $a$. Hence 
  the number $\delta$ may vary on different points in $E$.

  In the case of uniform continuity, we choose the number $\delta$ depending on 
  only the number $\epsilon$, and it should work for every point in $E$. In 
  other words, for every $\epsilon > 0$, there exists an greatest lower bound of 
  the set of $\delta$ for a function $f: E \to \mathbb{R}$ to be continuous on 
  each point $a \in E$.
\end{remark}
\begin{remark}[A sufficient condition of the negation of uniform continuity]
  If the function $f: E \to \mathbb{R}$ is unbounded in every neighborhood of a 
  fixed point $x_0 \in E$, then it is not uniformly continuous on $E$. If $E$ 
  consists of open intervals, then $x_0$ can be an endpoint of $E$ since $x_0$ 
  is a limit point of $E$ and every neighborhood $U(x_0)$ contains infinite 
  points of $E$.
\end{remark}
\begin{remark}[Another sufficient and necessary condition of the negation of uniform continuity]
  For a function $f: E \to \mathbb{R}$, if there exist a sequence $\lbrace x_n' 
  \rbrace$ and $\lbrace x_n'' \rbrace$ such that $\lim_{n \to +\infty} (x_n' - 
  x_n'') = 0$ but $(f(x_n') - f(x_n''))$ 不收敛于$0$.
\end{remark}

\begin{theorem}[The Contor-Heine theorem on uniform continuity]
  A function that is continuous on a closed interval is uniformly continuous on 
  that interval.
\end{theorem}
\begin{proof}
  Hint
\end{proof}

\subsection{Monotonic continuous functions}

\begin{proposition}
  A continuous mapping $f:E \to \mathbb{R}$ of a closed interval 
  $E = \lbrack a,b \rbrack$ into $\mathbb{R}$ is injective if and only if the 
  function $f$ is strictly monotonic on $\lbrack a,b \rbrack$.
\end{proposition}

\begin{proposition}
  Each strictly monotonic function $f:X \to \mathbb{R}$ defined on a numerical 
  set $X \subset \mathbb{R}$ has an inverse $f^{-1}:Y \to \mathbb{R}$ defined on 
  the set $Y = f(X)$ of values of $f$, and has the same kind of monotonicity on 
  $Y$ that $f$ has on $X$.
\end{proposition}

\begin{proposition}
  The discontinuities of a function $f:E \to \mathbb{R}$ that is monotonic on 
  the set $E \subset \mathbb{R}$ can be only discontinuities of first kind.
\end{proposition}

\begin{corollary}
  If $a$ is a point of discontinuity of a monotonic function $f:E \to \mathbb{R}$, 
  then at least one of the limits $\lim_{E \owns x \to a^+} = f(a^+)$ or 
  $\lim_{E \owns x \to a^-} = f(a^-)$ exists, and strict inequality holds in at 
  least one of the inequalities $f(a^-) \leq f(a) \leq f(a^+)$ when $f$ is 
  nondecreasing and $f(a^-) \geq f(a) \geq f(a^+)$ when $f$ is nonincreasing. 
  The function assumes no values in the open interval defined by the strict 
  inequality. Open intervals of this kind determined by different points of 
  discontinuity have no points in common.
\end{corollary}

\begin{corollary}
  The set of points of discontinuity of a monotonic function is at most 
  countable.
\end{corollary}

\begin{proposition}[A Criterion for Continuity of a Monotonic Function]
  A monotonic function $f:E \to \mathbb{R}$ defined on a closed interval 
  $E = \lbrack a,b \rbrack$ is continuous if and only if its set of values 
  $f(E)$ is the closed interval with endpoints $f(a)$ and $f(b)$.
\end{proposition}

\begin{theorem}[The Inverse Function Theorem]
  A function $f:X \to \mathbb{R}$ that is strictly monotonic on a set $X \subset 
  \mathbb{R}$ has an inverse $f^{-1}: Y \to \mathbb{R}$ defined on the set $Y = 
  f(X)$ of values of $f$, and has the same kind of monotonicity on $Y$ that $f$ 
  has on $X$.

  If in addition $X$ is a closed interval $\lbrack a,b \rbrack$ and $f$ is 
  continuous on $X$, then the set $Y = f(X)$ is the closed interval with 
  endpoints $f(a)$ and $f(b)$ and the function $f^{-1}: Y \to \mathbb{R}$ is 
  continuous on it.
\end{theorem}

\end{document}