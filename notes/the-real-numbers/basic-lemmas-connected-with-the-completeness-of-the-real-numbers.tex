\documentclass[onecolumn]{ctexart}
\usepackage[utf8]{inputenc}
\usepackage{amsmath}
\usepackage{amssymb}
\usepackage{amsthm}
\usepackage{geometry}
\usepackage{graphicx}
\usepackage{float}
\usepackage{xcolor}
\usepackage{listings}
\usepackage{indentfirst}
\usepackage{bm}
\usepackage{tikz}
\usetikzlibrary{shapes,arrows}
\geometry{a4paper,scale=0.8}

\newtheorem{definition}{Definition}
\newtheorem{theorem}{Theorem}
\newtheorem{proposition}{Proposition}
\newtheorem{lemma}{Lemma}
\newtheorem{corollary}{Corollary}
\newtheorem{remark}{Remark}
\newtheorem{example}{Example}

\title{Notes of "Basic Lemmas Connected with the Completeness of the Real Numbers"}
\author{Jinxin Wang}
\date{}

\begin{document}

\maketitle

\section{Overview}
\begin{itemize}
  \item The Nested Interval Lemma
  \begin{itemize}
    \item Definition: A sequence of elements of a set
    \item Definition: A sequence of nested intervals
    \item Lemma: The Nested Interval Lemma
  \end{itemize}
  \item The Finite Covering Lemma
  \begin{itemize}
    \item Definition: A cover of a set
    \item Lemma: The Finite Covering Lemma
  \end{itemize}
  \item The Limit Point Lemma
  \begin{itemize}
    \item Definition: A limit point of a set
    \item Lemma: The Limit Point Lemma
  \end{itemize}
\end{itemize}

\section{The Nested Interval Lemma (Cauchy-Cantor Principle)}

\begin{definition}[A Sequence of Elements of a Set]
  A function $f: \mathbb{N} \to X$ of a natural-number argument is called a 
  sequence or, more fully, a sequence of elements of $X$.
\end{definition}

\begin{definition}[A Sequence of Nested Intervals]
  Let $X_1, X_2, \ldots, X_n, \ldots$ be a sequence of sets. If $X_1 \supset X_2 
  \supset \cdots \supset X_n \supset \cdots$, that is $X_n \supset X_{n+1}$ for 
  all $n \in \mathbb{N}$, we say the sequence is nested.
\end{definition}

\begin{lemma}[The Nested Interval Lemma, or Cauchy-Cantor Principle]
  For any nested sequence $I_1 \supset I_2 \supset \cdots \supset I_n \supset 
  \cdots$ of closed intervals, there exists a point $c \in \mathbb{R}$ belonging 
  to all of these intervals.

  If in addition it is known that for any $\epsilon > 0$ there is an interval 
  $I_k$ whose length $|I_k|$ is less than $\epsilon$, then $c$ is the unique 
  point common to all the intervals.
\end{lemma}
\begin{proof}
  (TODO)
\end{proof}
\begin{remark}
  Notice the sequence of nested intervals are closed intervals. If they are open 
  intervals (TODO)
\end{remark}

\section{The Finite Covering Lemma (Borel-Lebesgue Principle)}

\begin{definition}[A Cover of a Set]
  A system $S = \lbrace X \rbrace$ of sets $X$ is said to cover a set $Y$ if $Y 
  \subset \bigcup_{X \in S} X$, that is, if every element $y \in Y$ belongs to 
  at least one of the sets $X$ in the system $S$.
\end{definition}

\begin{lemma}[The Finite Covering Lemma, or Borel-Lebesgue Principle]
  Every system of open intervals covering a closed interval containing a finite 
  subsystem that covers the closed interval.
\end{lemma}
\begin{proof}
  (TODO)
\end{proof}

\section{The Limit Point Lemma (Bolzano-Weierstrass Principle)}

\begin{definition}[A Limit Point of a Set]
  A point $p \in \mathbb{R}$ is a limit point of the set $X \subset \mathbb{R}$ 
  if every neighborhood of the point contains an infinite subset of $X$.
\end{definition}
\begin{remark}
  An equivalent condition is that every neighborhood of $p$ contains at least 
  one point of $X$ different from $p$ itself.
\end{remark}
\begin{remark}
  The concept of a limit point is 相对的. It is clear from the definition that we 
  must specify the set of a limit point. A limit point of a set $X_1 \subset 
  \mathbb{R}$ might not be a limit point in terms of another set $X_2 \subset 
  \mathbb{R}$. For example, a finite point set $X \subset \mathbb{R}$ can never 
  have a limit point since it does not have any infinite subset.
\end{remark}

\begin{lemma}[The Limit Point Principle, or Bolzano-Weierstrass Principle]
  Every bounded infinite set of real numbers has at least one limit point.
\end{lemma}
\begin{proof}
  (TODO)
\end{proof}

\end{document}