\documentclass[onecolumn]{ctexart}
\usepackage[utf8]{inputenc}
\usepackage{amsmath}
\usepackage{amssymb}
\usepackage{amsthm}
\usepackage{mathtools}
\usepackage{geometry}
\usepackage{graphicx}
\usepackage{float}
\usepackage{xcolor}
\usepackage{listings}
\usepackage{indentfirst}
\usepackage{bm}
\usepackage{tikz}
\usetikzlibrary{shapes,arrows}
\geometry{a4paper,scale=0.8}

\newtheorem{definition}{Definition}
\newtheorem{theorem}{Theorem}
\newtheorem{proposition}{Proposition}
\newtheorem{lemma}{Lemma}
\newtheorem{corollary}{Corollary}
\newtheorem{remark}{Remark}
\newtheorem{example}{Example}

\title{Notes of "The Axiom System and Some General Properties of the Set of Real Numbers"}
\author{Jinxin Wang}
\date{}

\begin{document}

\maketitle

\section{Overview}
\begin{itemize}
  \item Definition of the Set of Real Numbers
  \item Some General Algebraic Properties of Real Numbers
  \item The Completeness Axiom and the Existence of a Least Upper (or Greatest Lower) Bound of a Set of Numbers
  \begin{itemize}
    \item Definition: A bounded-above/bounded below set
    \item Definition: A bounded set
    \item Definition: The maximal/minimal element of a set
    \begin{itemize}
      \item Remark: The formal expression of the definition
      \item Remark: The uniqueness of the maximal/minimal element of a set
      \item Remark: The existence of the maximal/minimal element of a set
    \end{itemize}
    \item Definition: The least upper bound and the greatest lower bound
    \begin{itemize}
      \item Remark: The formal express of the definition
    \end{itemize}
    \item Lemma: The least upper bound principle
  \end{itemize}
\end{itemize}

\section{Definition of the Set of Real Numbers}

\section{Some General Algebraic Properties of Real Numbers}

\section{The Completeness Axiom and the Existence of a Least Upper (or Greatest Lower) Bound of a Set of Numbers}

\begin{definition}[A Bounded-Above/Bounded-Below Set]
  A set $X \subset \mathbb{R}$ is said to be bounded above (resp. bounded below) 
  if there exists a number $c \in \mathbb{R}$ such that $x \leq c$ (resp. $c 
  \leq x$) for all $x \in X$.
\end{definition}

\begin{definition}[A Bounded Set]
  A set that is bounded both above and below is called bounded.
\end{definition}

\begin{definition}[The Maximal/Minimal Element of a Set]
  An element $a \in X$ is called the largest or maximal (resp. the smallest or 
  minimal) element of $X$ if $x \leq a$ (resp. $a \leq x$) for all $x \in X$
\end{definition}
\begin{remark}[The Formal Expression of the Definition]
  We can write the above definition in a formal expression:
  \[
    (a = \max X) \coloneqq (a \in X \wedge \forall x \in X (x \leq a))
  \]
  \[
    (a = \min X) \coloneqq (a \in X \wedge \forall x \in X (a \leq x))
  \]
\end{remark}
\begin{remark}[The Uniqueness of the Maximal/Minimal Element of a Set]
  
\end{remark}
\begin{remark}[The Existence of the Maximal/Minimal Element of a Set]
  Not every set, or not even every bounded set, has a maximal or minimal element.
  Example: $I = \lbrace x \mid 0 \leq x < 1 \rbrace$ does not have a maximal 
  element. In my opinion it is due to the 稠密性 of real numbers.
\end{remark}

\begin{definition}[The Least Upper Bound]
  The minimal number that bounds a set $X \subset \mathbb{R}$ above is called 
  the least upper bound of $X$ and denoted $\sup X$ (the supremum of X) or 
  $\sup_{x \in X} x$.
\end{definition}
\begin{remark}
  \[
    (s = \sup X) \coloneqq ((\forall x \in X (x \leq s)) \wedge (\forall s' < s \exists x' \in X (s' < x')))
  \]
\end{remark}

\begin{definition}[The Greatest Lower Bound]
  Similarly, the greatest lower bound of $X$, denoted $\inf X$ (the infimum of X) 
  or $\inf_{x \in X} x$, is defined as
  \[
    (i = \inf X) \coloneqq ((\forall x \in X (i \leq x)) \wedge (\forall i' > i \exists x' \in X (x' < i')))
  \]
\end{definition}

\begin{lemma}[The Least Upper Bound Principle]
  Every nonempty set of real numbers that is bounded from above has a unique 
  least upper bound.
\end{lemma}
\begin{proof}
  (TODO)
\end{proof}

\begin{lemma}[The Greatest Lower Bound Principle]
  \[
    (\textnormal{X is nonempty and bounded below}) \Leftarrow (\exists! \inf X)
  \]
\end{lemma}



\end{document}